\newpage
\begin{center}
    \fontsize{12}{11}\selectfont
    \textbf{DECLARAȚIE DE ORIGINALITATE}
    \\[4\baselineskip]
    \end{center}
    
    \renewcommand{\baselinestretch}{1.8}
    \raggedright
        \fontsize{11}{11}\selectfont
    Subsemnatul \textcolor{red}{[\textit{PRENUMELE ȘI NUMELE CANDIDATULUI}]}, student la specializarea \textcolor{red}{[\textit{DENUMIREA OFICIALĂ 
    A SPECIALIZĂRII}]} din cadrul Facultății de Automatică, Calculatoare și Electronică a Universității din Craiova, certific prin 
    prezenta că am luat la cunoştinţă de cele prezentate mai jos şi că îmi asum, în acest context, originalitatea 
    proiectului meu de licenţă: 
        \begin{itemize}
            \setlength\itemsep{-0.5em}
            \item cu titlul \textcolor{red}{[\textit{TITLUL LUCRĂRII}]}, 
            \item coordonată de \textcolor{red}{[\textit{TITLUL ȘTIINȚIFIC, PRENUMELE ȘI NUMELE COORDONATORULUI}]}, 
            \item prezentată în sesiunea \textcolor{red}{[\textit{LUNA ȘI ANUL SESIUNII DE LICENȚĂ}]}.
        \end{itemize}
    
        La elaborarea proiectului de licenţă, se consideră plagiat una dintre următoarele acţiuni: 
        \begin{itemize}
            \setlength\itemsep{-0.5em}
            \item reproducerea exactă a cuvintelor unui alt autor, dintr-o altă lucrare, în limba română sau prin traducere dintr-o altă limbă, dacă se omit ghilimele şi referinţa precisă, 
            \item redarea cu alte cuvinte, reformularea prin cuvinte proprii sau rezumarea ideilor din alte lucrări, dacă nu se indică sursa bibliografică, 
            \item prezentarea unor date experimentale obţinute sau a unor aplicaţii realizate de alţi autori fără menţionarea corectă a acestor surse, 
            \item însuşirea totală sau parţială a unei lucrări în care regulile de mai sus sunt respectate, dar care are alt autor. 
        \end{itemize}
        Pentru evitarea acestor situaţii neplăcute se recomandă:
        \begin{itemize}
            \setlength\itemsep{-0.5em}
            \item plasarea între ghilimele a citatelor directe şi indicarea referinţei într-o listă corespunzătoare la sfărşitul lucrării, 
            \item indicarea în text a reformulării unei idei, opinii sau teorii şi corespunzător în lista de referinţe a sursei originale de la care s-a făcut preluarea, 
            \item precizarea sursei de la care s-au preluat date experimentale, descrieri tehnice, figuri, imagini, statistici, tabele et caetera, 
            \item precizarea referinţelor poate fi omisă dacă se folosesc informaţii sau teorii arhicunoscute, a căror paternitate este unanim cunoscută și acceptată.
        \end{itemize}
        Data,  \hfill Semnătura candidatului, 